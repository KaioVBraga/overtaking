%---------------------------------------------------------------------
% Relatório Técnico: Controle Autônomo no TORCS usando Lógica Fuzzy
% Versão corrigida para o template da Sociedade Brasileira de Computação (SBC)
%---------------------------------------------------------------------
\documentclass[12pt]{article}

% --- Pacotes e Configurações ---
\usepackage{sbc-template}
\usepackage[utf8]{inputenc}
\usepackage[T1]{fontenc}
\usepackage[brazil]{babel}
\usepackage{graphicx}
\usepackage{url}
\usepackage{amsmath}
\usepackage{booktabs}
\usepackage{filecontents}
\usepackage{enumitem}
\usepackage{longtable} % Para tabelas que podem quebrar a página
\usepackage{caption}   % Para customizar legendas

%---------------------------------------------------------------------
% Bibliografia embutida (arquivo .bib gerado automaticamente)
%---------------------------------------------------------------------
\begin{filecontents*}{referencias.bib}
@misc{torcs2005,
  author = {Wymann, Bernhard and Espi{\'e}, Eric and Guionneau, Christophe and Dimitrakakis, Christos and Moday, Manolo and Lagarrigue, Jean-Philippe},
  title  = {{TORCS}, The Open Racing Car Simulator},
  howpublished = {\url{http://torcs.sourceforge.net}},
  year   = {2005},
  note   = {Acessado em: 21 ago. 2025}
}

@misc{skfuzzy,
  author = {Warner, Joshua and Newville, Matthew and others},
  title  = {scikit-fuzzy},
  year   = {2019},
  publisher = {GitHub},
  journal = {GitHub repository},
  howpublished = {\url{https://github.com/scikit-fuzzy/scikit-fuzzy}},
  note = {Acessado em: 21 ago. 2025}
}

@article{zadeh1965,
  author  = {Zadeh, Lotfi A.},
  title   = {Fuzzy sets},
  journal = {Information and Control},
  volume  = {8},
  number  = {3},
  pages   = {338--353},
  year    = {1965},
  doi     = {10.1016/S0019-9958(65)90241-X}
}

@article{mamdani1975,
  author    = {Mamdani, Ebrahim H. and Assilian, Sedrak},
  title     = {An experiment in linguistic synthesis with a fuzzy logic controller},
  journal   = {International Journal of Man-Machine Studies},
  volume    = {7},
  number    = {1},
  pages     = {1--13},
  year      = {1975},
  publisher = {Elsevier}
}
\end{filecontents*}

% --- Metadados do Artigo ---
\title{Relat\'orio T\'ecnico: Controlador Fuzzy para Ve\'iculos Aut\^onomos no TORCS}
\author{Rafael de Oliveira Vargas\inst{1} \and Kaio Vinycius Braga dos Santos\inst{1}}
\address{
    Departamento de Ci\^encia da Computa\c{c}\~ao -- Universidade Federal de Juiz de Fora (UFJF)\\
    Juiz de Fora -- MG -- Brazil\\
    \email{rafael.vargas@estudante.ufjf.br, kaio.braga@estudante.ufjf.br}
}

% --- Início do Documento ---
\begin{document}

\maketitle

% --- Resumo e Abstract ---
\begin{resumo}
Este trabalho descreve o desenvolvimento e a integra\c{c}\~ao de um controlador baseado em l\'ogica fuzzy para a condu\c{c}\~ao de um ve\'iculo no ambiente de simula\c{c}\~ao TORCS. O artefato, implementado em Python, organiza-se em arquivos-chave respons\'aveis pela interpreta\c{c}\~ao de sensores, gera\c{c}\~ao de a\c{c}\~oes e comunica\c{c}\~ao com o servidor TORCS. As estruturas de infer\^encia utilizam a biblioteca \texttt{scikit-fuzzy} para definir antecedentes e consequentes, fun\c{c}\~oes de pertin\^encia triangulares e sistemas de controle com \texttt{ControlSystem} e \texttt{ControlSystemSimulation}. O relat\'orio foca nos m\'odulos de controle que comp\~oem a tomada de decis\~ao, descrevendo como os modelos fuzzy s\~ao constru\'idos e utilizados para produzir os sinais de dire\c{c}\~ao, acelera\c{c}\~ao, freio e sele\c{c}\~ao de marcha.
\end{resumo}

\begin{abstract}
This work describes the development and integration of a fuzzy logic-based controller for vehicle driving in the TORCS simulation environment. The artifact, implemented in Python, is organized into key files responsible for sensor interpretation, action generation, and communication with the TORCS server. The inference structures use the \texttt{scikit-fuzzy} library to define antecedents and consequents, triangular membership functions, and control systems with \texttt{ControlSystem} and \texttt{ControlSystemSimulation}. The report focuses on the control modules that make decisions, describing how fuzzy models are constructed and used to produce steering, acceleration, braking, and gear shift signals.
\end{abstract}

% --- Seções Principais ---
\section{Introdu\c{c}\~ao}
A aplica\c{c}\~ao de l\'ogica fuzzy ao controle de ve\'iculos possibilita a incorpora\c{c}\~ao de heur\'isticas humanas e regras lingu\'isticas em controladores robustos capazes de lidar com medi\c{c}\~oes ruidosas e situa\c{c}\~oes n\~ao lineares \cite{zadeh1965, mamdani1975}. O simulador TORCS (The Open Racing Car Simulator) \cite{torcs2005} oferece uma plataforma consolidada para validar controladores aut\^onomos em cen\'arios reprodut\'iveis. O presente trabalho descreve o projeto e a implementa\c{c}\~ao de um controlador fuzzy implementado em Python, organizado em m\'odulos que se integram ao servidor TORCS atrav\'es de um cliente socket.

A quest\~ao-problema definida para este trabalho \emph{\'e}: ``Projetar um controlador Fuzzy e integrar o mesmo ao TORCS, fazendo o gerenciamento do controle de acelera\c{c}\~ao, freio, dire\c{c}\~ao e marcha com o objetivo de conseguir ter um ve\'iculo aut\^onomo que se comporte bem a diferentes tra\c{c}ados de pista''.

O objetivo geral do trabalho foi implementar e avaliar um controlador fuzzy no ambiente TORCS que gerencie dire\c{c}\~ao, acelera\c{c}\~ao, freio e marcha. Entre os objetivos espec\'ificos est\~ao a integra\c{c}\~ao dos sensores do TORCS ao controlador fuzzy, a defini\c{c}\~ao de regras fuzzy para as sa\'idas de controle e a avalia\c{c}\~ao qualitativa do comportamento do ve\'iculo em diferentes tipos de pista.

\section{Materiais e M\'etodos}
A implementa\c{c}\~ao analisada est\'a organizada em um n\'ucleo de arquivos que coordenam a tomada de decis\~ao e a comunica\c{c}\~ao com o simulador. O arquivo \texttt{torcs\_driver.py} cont\'em a classe \texttt{TorcsDriver}, que orquestra o ciclo de controle e armazena o estado interno. O arquivo \texttt{torcs\_client.py} implementa o cliente socket respons\'avel por estabelecer a comunica\c{c}\~ao com o servidor TORCS. O diret\'orio \texttt{Interpretation} inclui o m\'odulo \texttt{track.py}, que constr\'oi um classificador fuzzy para curvas, enquanto o diret\'orio \texttt{Actions} abriga os m\'odulos \texttt{acceleration.py}, \texttt{gear.py} e \texttt{steering.py}, que implementam os controladores fuzzy para seus respectivos atuadores.

As ferramentas utilizadas incluem o simulador TORCS, a biblioteca \texttt{scikit-fuzzy} \cite{skfuzzy} para a constru\c{c}\~ao dos sistemas de controle, e o ambiente Python 3.x para integrar os componentes. A implementa\c{c}\~ao estruturou vari\'aveis lingu\'isticas para entradas como severidade da curva e velocidade, e para sa\'idas como inten\c{c}\~ao de acelera\c{c}\~ao/freio e sinal de troca de marcha. As regras foram formuladas no estilo SE-ENT\~AO e encapsuladas em inst\^ancias de \texttt{ControlSystem}. O método de defuzzificação utilizado em todos os sistemas é o Centroide de Área (CoA), que é o padrão da biblioteca.

\section{Resultados e Discuss\~ao}
A avalia\c{c}\~ao do comportamento do controlador foi realizada de forma qualitativa. Em pistas ovais, o ve\'iculo apresentou comportamento consistente. Em pistas com tra\c{c}ados mais complexos, o controlador consegue seguir a pista, mas mostra limita\c{c}\~oes: as frenagens s\~ao, por vezes, bruscas e a trajet\'oria tende a permanecer centrada, em vez de seguir o tra\c{c}ado \emph{otimizado} para tempo de volta. Em intera\c{c}\~oes com outros ve\'iculos, o sistema evita colis\~oes frontais, contudo carece de percep\c{c}\~ao lateral, pois sensores de 360 graus n\~ao foram implementados.

A discuss\~ao qualitativa indica que a escolha por fun\c{c}\~oes de pertin\^encia triangulares e regras program\'aticas fornece previsibilidade, mas imp\~oe rigidez na resposta. A parametriza\c{c}\~ao manual das fun\c{c}\~oes mostrou-se sens\'ivel ao tipo de pista, o que evidencia a necessidade de m\'etodos de otimiza\c{c}\~ao. Em suma, o sistema \'e funcional em cen\'arios controlados, com potencial de expans\~ao para cen\'arios mais complexos.

\subsection{Definição das Variáveis Fuzzy e Funções de Pertinência}
A seguir, são detalhadas as variáveis linguísticas (antecedentes e consequentes) e suas respectivas funções de pertinência para cada sistema de controle fuzzy. Todas as funções de pertinência são do tipo triangular (\texttt{trimf}), exceto quando especificado.

A Tabela \ref{tab:track_vars} detalha as variáveis utilizadas pelo sistema classificador de curvas, responsável por interpretar a geometria da pista à frente.
\begin{longtable}{llll}
\toprule
\textbf{Variável} & \textbf{Descrição} & \textbf{Universo} & \textbf{Termos Linguísticos e Definições} \\
\midrule
\endfirsthead
\multicolumn{4}{c}%
{{\bfseries \tablename\ \thetable{} -- continuação}} \\
\toprule
\textbf{Variável} & \textbf{Descrição} & \textbf{Universo} & \textbf{Termos Linguísticos e Definições} \\
\midrule
\endhead
\bottomrule
\endfoot
\texttt{center} & Distância do sensor frontal & [0, 100] & \texttt{very\_close}: [0, 0, 30] \\
(Antecedente) & & & \texttt{close}: [20, 40, 60] \\
& & & \texttt{far}: [50, 100, 100] \\
\midrule
\texttt{side} & Diferença entre sensores laterais & [0, 100] & \texttt{small}: [0, 0, 20] \\
(Antecedente) & & & \texttt{medium}: [10, 30, 50] \\
& & & \texttt{large}: [40, 100, 100] \\
\midrule
\texttt{turn\_severity} & Severidade da curva & [0, 1] & \texttt{straight}: [0.0, 0.0, 0.2] \\
(Consequente) & & & \texttt{long\_turn}: [0.1, 0.3, 0.5] \\
& & & \texttt{medium}: [0.35, 0.55, 0.75] \\
& & & \texttt{sharp}: [0.6, 1.0, 1.0] \\
\caption{\textbf{Variáveis do Classificador de Curva (\texttt{track.py})}} \label{tab:track_vars}
\end{longtable}

As variáveis que definem a intenção de aceleração e frenagem são apresentadas na Tabela \ref{tab:accel_vars}.
\begin{longtable}{llll}
\toprule
\textbf{Variável} & \textbf{Descrição} & \textbf{Universo} & \textbf{Termos Linguísticos e Definições} \\
\midrule
\endfirsthead
\multicolumn{4}{c}%
{{\bfseries \tablename\ \thetable{} -- continuação}} \\
\toprule
\textbf{Variável} & \textbf{Descrição} & \textbf{Universo} & \textbf{Termos Linguísticos e Definições} \\
\midrule
\endhead
\bottomrule
\endfoot
\texttt{turn\_severity} & Severidade da curva & [0, 1] & \texttt{straight}: [0.0, 0.0, 0.25] \\
(Antecedente) & & & \texttt{long}: [0.15, 0.3, 0.5] \\
& & & \texttt{medium}: [0.35, 0.55, 0.75] \\
& & & \texttt{sharp}: [0.6, 1.0, 1.0] \\
\midrule
\texttt{speed} & Velocidade do veículo (km/h) & [0, 350] & \texttt{low}: [0, 0, 60] \\
(Antecedente) & & & \texttt{mid}: [40, 120, 200] \\
& & & \texttt{high}: [150, 250, 350] \\
\midrule
\texttt{intention} & Intenção de acelerar ou frear & [-1, 1] & \texttt{strong\_brake}: [-1.0, -1.0, -0.5] \\
(Consequente) & & & \texttt{brake}: [-0.8, -0.4, -0.1] \\
& & & \texttt{coast}: [-0.2, 0.0, 0.2] \\
& & & \texttt{gentle\_acc}: [0.1, 0.4, 0.7] \\
& & & \texttt{full\_acc}: [0.5, 1.0, 1.0] \\
\caption{\textbf{Variáveis do Controlador de Aceleração/Freio (\texttt{accelaration.py})}} \label{tab:accel_vars}
\end{longtable}

A Tabela \ref{tab:steering_vars} descreve as variáveis do controlador de agressividade da direção, que ajusta a suavidade dos movimentos do volante.
\begin{longtable}{llll}
\toprule
\textbf{Variável} & \textbf{Descrição} & \textbf{Universo} & \textbf{Termos Linguísticos e Definições} \\
\midrule
\endfirsthead
\multicolumn{4}{c}%
{{\bfseries \tablename\ \thetable{} -- continuação}} \\
\toprule
\textbf{Variável} & \textbf{Descrição} & \textbf{Universo} & \textbf{Termos Linguísticos e Definições} \\
\midrule
\endhead
\bottomrule
\endfoot
\texttt{speed} & Velocidade do veículo (km/h) & [0, 350] & \texttt{low}: [0, 0, 60] \\
(Antecedente) & & & \texttt{mid}: [40, 120, 200] \\
& & & \texttt{high}: [150, 250, 350] \\
\midrule
\texttt{severity} & Severidade da curva & [0, 1] & \texttt{low}: [0.0, 0.0, 0.3] \\
(Antecedente) & & & \texttt{medium}: [0.2, 0.5, 0.8] \\
& & & \texttt{high}: [0.6, 1.0, 1.0] \\
\midrule
\texttt{dist\_to\_turn} & Distância estimada para a curva & [0, 100] & \texttt{very\_close}: [0, 0, 20] \\
(Antecedente) & & & \texttt{close}: [10, 30, 50] \\
& & & \texttt{far}: [40, 100, 100] \\
\midrule
\texttt{aggressiveness} & Fator de suavização da direção & [0.1, 1.0] & \texttt{gentle}: [0.1, 0.1, 0.4] \\
(Consequente) & & & \texttt{normal}: [0.3, 0.5, 0.7] \\
& & & \texttt{aggressive}: [0.6, 1.0, 1.0] \\
\caption{\textbf{Variáveis do Controlador de Agressividade da Direção (\texttt{steering.py})}} \label{tab:steering_vars}
\end{longtable}

Finalmente, a Tabela \ref{tab:gear_vars} mostra a estrutura complexa de variáveis para o sistema de controle de marchas, que considera múltiplos fatores para uma troca eficiente.
\begin{longtable}{llll}
\toprule
\textbf{Variável} & \textbf{Descrição} & \textbf{Universo} & \textbf{Termos Linguísticos e Definições} \\
\midrule
\endfirsthead
\multicolumn{4}{c}%
{{\bfseries \tablename\ \thetable{} -- continuação}} \\
\toprule
\textbf{Variável} & \textbf{Descrição} & \textbf{Universo} & \textbf{Termos Linguísticos e Definições} \\
\midrule
\endhead
\bottomrule
\endfoot
\texttt{intention} & Intenção de acelerar ou frear & [-1, 1] & \texttt{braking}: [-1, -1, -0.2] \\
(Antecedente) & & & \texttt{coast}: [-0.3, 0, 0.3] \\
& & & \texttt{accel}: [0.1, 1, 1] \\
\midrule
\texttt{rpm} & Rotações por minuto do motor & [0, 10000] & \texttt{very\_low}: [0, 0, 2000] \\
(Antecedente) & & & \texttt{low}: [1000, 3000, 4000] \\
& & & \texttt{mid}: [3000, 5000, 7000] \\
& & & \texttt{high}: [6000, 8000, 10000] \\
& & & \texttt{very\_high}: [8000, 10000, 10000] \\
\midrule
\texttt{speed} & Velocidade do veículo (km/h) & [0, 350] & \texttt{low}: [0, 0, 60] \\
(Antecedente) & & & \texttt{mid}: [40, 120, 200] \\
& & & \texttt{high}: [150, 250, 350] \\
\midrule
\texttt{gear\_in} & Marcha atual & [1, 6] & \texttt{low}: ZMF(1, 3) \\
(Antecedente) & & & \texttt{mid}: PIMF(2, 3, 4, 5) \\
& & & \texttt{high}: SMF(4, 6) \\
\midrule
\texttt{severity} & Severidade da curva & [0, 1] & \texttt{low}: [0.0, 0.0, 0.3] \\
(Antecedente) & & & \texttt{medium}: [0.2, 0.5, 0.8] \\
& & & \texttt{high}: [0.6, 1.0, 1.0] \\
\midrule
\texttt{gear\_adj} & Ajuste de marcha & [-1, 1] & \texttt{down}: [-1, -1, -0.4] \\
(Consequente) & & & \texttt{keep}: [-0.3, 0, 0.3] \\
& & & \texttt{up}: [0.4, 1, 1] \\
\caption{\textbf{Variáveis do Controlador de Troca de Marchas (\texttt{gear.py})}} \label{tab:gear_vars}
\end{longtable}

\subsection{Análise das Regras da Lógica Fuzzy no Piloto de IA do TORCS}
Este documento descreve o conjunto completo de regras de lógica fuzzy que governam o processo de tomada de decisão do piloto de IA \textbf{PilotoNebuloso}. As regras são definidas em linguagem natural para esclarecer a lógica por trás do comportamento do bot em diferentes cenários de pilotagem. Elas estão agrupadas de acordo com o sistema de controle específico ao qual pertencem.

\subsubsection{Interpretação da Pista: Classificador de Curva (\texttt{track.py})}
O objetivo deste sistema é analisar a pista à frente e determinar a \textbf{severidade de uma curva futura}. Ele utiliza os sensores de pista frontais e laterais como entradas.
\begin{itemize}[label=\textbullet, leftmargin=*]
    \item \textbf{SE} a pista à frente está \texttt{far} \textbf{E} a diferença lateral é \texttt{small}, \textbf{ENTÃO} a pista é \texttt{straight}.
    \item \textbf{SE} a pista à frente está \texttt{far} \textbf{E} a diferença lateral é \texttt{medium}, \textbf{ENTÃO} é uma \texttt{long\_turn}.
    \item \textbf{SE} a pista à frente está \texttt{far} \textbf{E} a diferença lateral é \texttt{large}, \textbf{ENTÃO} é uma \texttt{long\_turn}.
    \item \textbf{SE} a pista à frente está \texttt{close} \textbf{E} a diferença lateral é \texttt{small}, \textbf{ENTÃO} é uma \texttt{long\_turn}.
    \item \textbf{SE} a pista à frente está \texttt{close} \textbf{E} a diferença lateral é \texttt{medium}, \textbf{ENTÃO} é uma curva \texttt{medium}.
    \item \textbf{SE} a pista à frente está \texttt{close} \textbf{E} a diferença lateral é \texttt{large}, \textbf{ENTÃO} é uma curva \texttt{sharp}.
    \item \textbf{SE} a pista à frente está \texttt{very\_close} \textbf{E} a diferença lateral é \texttt{small}, \textbf{ENTÃO} é uma curva \texttt{medium}.
    \item \textbf{SE} a pista à frente está \texttt{very\_close} \textbf{E} a diferença lateral é \texttt{medium}, \textbf{ENTÃO} é uma curva \texttt{sharp}.
    \item \textbf{SE} a pista à frente está \texttt{very\_close} \textbf{E} a diferença lateral é \texttt{large}, \textbf{ENTÃO} é uma curva \texttt{sharp}.
\end{itemize}

\subsubsection{Controlador de Aceleração e Freio (\texttt{accelaration.py})}
Este sistema determina a \textbf{intenção} primária do piloto (acelerar, frear ou deixar o carro rolar) com base na severidade da curva (do classificador) e na velocidade atual do carro.
\begin{itemize}[label=\textbullet, leftmargin=*]
    \item \textbf{Seções Retas}:
    \begin{itemize}
        \item \textbf{SE} a curva é \texttt{straight} \textbf{E} a velocidade é \texttt{low}, \textbf{ENTÃO} a intenção é \texttt{full\_acc} (aceleração total).
        \item \textbf{SE} a curva é \texttt{straight} \textbf{E} a velocidade é \texttt{mid}, \textbf{ENTÃO} a intenção é \texttt{full\_acc}.
        \item \textbf{SE} a curva é \texttt{straight} \textbf{E} a velocidade é \texttt{high}, \textbf{ENTÃO} a intenção é \texttt{full\_acc}.
    \end{itemize}
    \item \textbf{Curvas Longas}:
    \begin{itemize}
        \item \textbf{SE} a curva é \texttt{long} \textbf{E} a velocidade é \texttt{low}, \textbf{ENTÃO} a intenção é \texttt{gentle\_acc} (aceleração suave).
        \item \textbf{SE} a curva é \texttt{long} \textbf{E} a velocidade é \texttt{mid}, \textbf{ENTÃO} a intenção é \texttt{coast} (deixar rolar).
        \item \textbf{SE} a curva é \texttt{long} \textbf{E} a velocidade é \texttt{high}, \textbf{ENTÃO} a intenção é \texttt{brake} (frear).
    \end{itemize}
    \item \textbf{Curvas Médias}:
    \begin{itemize}
        \item \textbf{SE} a curva é \texttt{medium} \textbf{E} a velocidade é \texttt{low}, \textbf{ENTÃO} a intenção é \texttt{coast}.
        \item \textbf{SE} a curva é \texttt{medium} \textbf{E} a velocidade é \texttt{mid}, \textbf{ENTÃO} a intenção é \texttt{brake}.
        \item \textbf{SE} a curva é \texttt{medium} \textbf{E} a velocidade é \texttt{high}, \textbf{ENTÃO} a intenção é \texttt{strong\_brake} (frear forte).
    \end{itemize}
    \item \textbf{Curvas Fechadas}:
    \begin{itemize}
        \item \textbf{SE} a curva é \texttt{sharp} \textbf{E} a velocidade é \texttt{low}, \textbf{ENTÃO} a intenção é \texttt{brake}.
        \item \textbf{SE} a curva é \texttt{sharp} \textbf{E} a velocidade é \texttt{mid}, \textbf{ENTÃO} a intenção é \texttt{strong\_brake}.
        \item \textbf{SE} a curva é \texttt{sharp} \textbf{E} a velocidade é \texttt{high}, \textbf{ENTÃO} a intenção é \texttt{strong\_brake}.
    \end{itemize}
\end{itemize}

\subsubsection{Controlador de Agressividade da Direção (\texttt{steering.py})}
Este sistema não calcula o ângulo de esterçamento diretamente. Em vez disso, ele determina \emph{quão agressivamente} aplicar o esterçamento calculado.
\begin{itemize}[label=\textbullet, leftmargin=*]
    \item \textbf{SE} a severidade da curva é \texttt{high} \textbf{E} a velocidade é \texttt{high}, \textbf{ENTÃO} a agressividade é \texttt{gentle}.
    \item \textbf{SE} a severidade da curva é \texttt{high} \textbf{E} a velocidade é \texttt{mid}, \textbf{ENTÃO} a agressividade é \texttt{gentle}.
    \item \textbf{SE} a severidade da curva é \texttt{high} \textbf{E} a velocidade é \texttt{low}, \textbf{ENTÃO} a agressividade é \texttt{normal}.
    \item \textbf{SE} a severidade da curva é \texttt{medium} \textbf{E} a distância para a curva é \texttt{far}, \textbf{ENTÃO} a agressividade é \texttt{normal}.
    \item \textbf{SE} a severidade da curva é \texttt{medium} \textbf{E} a distância para a curva é \texttt{close}, \textbf{ENTÃO} a agressividade é \texttt{gentle}.
    \item \textbf{SE} a severidade da curva é \texttt{medium} \textbf{E} a distância para a curva é \texttt{very\_close}, \textbf{ENTÃO} a agressividade é \texttt{gentle}.
    \item \textbf{SE} a severidade da curva é \texttt{low} \textbf{E} a distância para a curva é \texttt{far}, \textbf{ENTÃO} a agressividade é \texttt{aggressive}.
    \item \textbf{SE} a severidade da curva é \texttt{low} \textbf{E} a velocidade é \texttt{low}, \textbf{ENTÃO} a agressividade é \texttt{normal}.
    \item \textbf{SE} a severidade da curva é \texttt{low} \textbf{E} a velocidade é \texttt{high} \textbf{E} a distância para a curva é \texttt{close}, \textbf{ENTÃO} a agressividade é \texttt{normal}.
    \item \textbf{SE} a velocidade é \texttt{high} \textbf{E} a severidade da curva é \texttt{medium}, \textbf{ENTÃO} a agressividade é \texttt{gentle}.
    \item \textbf{SE} a velocidade é \texttt{high} \textbf{E} a distância para a curva é \texttt{close}, \textbf{ENTÃO} a agressividade é \texttt{gentle}.
\end{itemize}

\subsubsection{Controlador de Troca de Marchas (\texttt{gear.py})}
Este é o sistema mais complexo, gerenciando as trocas de marcha. A saída é uma decisão para subir (\texttt{up}), reduzir (\texttt{down}) ou manter (\texttt{keep}) a marcha atual.
\begin{itemize}[label=\textbullet, leftmargin=*]
    \item \textbf{Prioridade 1: Segurança nas Curvas}
    \begin{itemize}
        \item \textbf{SE} a severidade da curva é \texttt{high} \textbf{E} a velocidade é \texttt{low}, \textbf{ENTÃO} reduza (\texttt{down}).
        \item \textbf{SE} a severidade da curva é \texttt{high} \textbf{E} a velocidade é \texttt{mid} \textbf{E} a intenção é \texttt{braking}, \textbf{ENTÃO} reduza (\texttt{down}).
        \item \textbf{SE} a severidade da curva é \texttt{high} \textbf{E} a velocidade é \texttt{mid} \textbf{E} a intenção é \texttt{coast}, \textbf{ENTÃO} reduza (\texttt{down}).
    \end{itemize}
    \item \textbf{Prioridade 2: Frenagem}
    \begin{itemize}
        \item \textbf{SE} a intenção é \texttt{braking}, \textbf{ENTÃO} reduza (\texttt{down}).
    \end{itemize}
    \item \textbf{Prioridade 3: Lógica de Aceleração}
    \begin{itemize}
        \item \textbf{SE} o RPM é \texttt{very\_high}, \textbf{ENTÃO} suba (\texttt{up}).
        \item \textbf{SE} o RPM é \texttt{high} \textbf{E} a marcha atual \textbf{NÃO} é \texttt{high}, \textbf{ENTÃO} suba (\texttt{up}).
        \item \textbf{SE} a intenção é \texttt{accel} \textbf{E} o RPM é \texttt{mid}, \textbf{ENTÃO} mantenha (\texttt{keep}).
        \item \textbf{SE} a intenção é \texttt{accel} \textbf{E} o RPM é \texttt{high}, \textbf{ENTÃO} suba (\texttt{up}).
        \item \textbf{SE} a intenção é \texttt{accel} \textbf{E} o RPM é \texttt{low} \textbf{E} a marcha atual é \texttt{mid}, \textbf{ENTÃO} reduza (\texttt{down}) (para ganhar torque).
        \item \textbf{SE} a intenção é \texttt{accel} \textbf{E} o RPM é \texttt{low} \textbf{E} a marcha atual é \texttt{high}, \textbf{ENTÃO} reduza (\texttt{down}).
        \item \textbf{SE} a intenção é \texttt{accel} \textbf{E} o RPM é \texttt{very\_low}, \textbf{ENTÃO} reduza (\texttt{down}).
    \end{itemize}
    \item \textbf{Prioridade 4: Lógica de "Coasting"}
    \begin{itemize}
        \item \textbf{SE} a intenção é \texttt{coast} \textbf{E} o RPM é \texttt{mid}, \textbf{ENTÃO} mantenha (\texttt{keep}).
        \item \textbf{SE} a intenção é \texttt{coast} \textbf{E} o RPM é \texttt{high} \textbf{E} a marcha atual \textbf{NÃO} é \texttt{high}, \textbf{ENTÃO} suba (\texttt{up}).
        \item \textbf{SE} a intenção é \texttt{coast} \textbf{E} o RPM é \texttt{low} \textbf{E} a marcha atual é \texttt{high}, \textbf{ENTÃO} reduza (\texttt{down}).
    \end{itemize}
    \item \textbf{Prioridade 5: Troca Antecipada}
    \begin{itemize}
        \item \textbf{SE} a severidade da curva é \texttt{high} \textbf{E} a velocidade é \texttt{high} \textbf{E} a intenção é \texttt{accel}, \textbf{ENTÃO} reduza (\texttt{down}) (antecipando a necessidade de frear).
        \item \textbf{SE} a severidade da curva é \texttt{medium} \textbf{E} a intenção é \texttt{braking}, \textbf{ENTÃO} reduza (\texttt{down}).
    \end{itemize}
    \item \textbf{Condição Estável e Regras de Fallback}
    \begin{itemize}
        \item \textbf{SE} a intenção é \texttt{accel} \textbf{E} o RPM é \texttt{mid} \textbf{E} a velocidade é \texttt{high} \textbf{E} a marcha atual é \texttt{high}, \textbf{ENTÃO} mantenha (\texttt{keep}).
        \item \textbf{SE} a intenção é \texttt{coast} \textbf{E} o RPM é \texttt{mid} \textbf{E} a velocidade é \texttt{mid}, \textbf{ENTÃO} mantenha (\texttt{keep}).
        \item \textbf{SE} (condição de fallback: a intenção não é frear nem acelerar, o RPM é mid e a severidade da curva não é alta), \textbf{ENTÃO} mantenha (\texttt{keep}).
    \end{itemize}
\end{itemize}

\section{Conclus\~ao}
O desenvolvimento alcan\c{c}ou o objetivo principal de implementar e integrar um controlador fuzzy ao ambiente TORCS. Foram identificadas limita\c{c}\~oes que indicam dire\c{c}\~oes para trabalhos futuros: a aus\^encia de sensores 360 graus restringe a percep\c{c}\~ao lateral; o sistema n\~ao persegue um tra\c{c}ado \emph{otimizado}; e a parametriza\c{c}\~ao manual exige ajuste para cada pista. Recomendam-se investiga\c{c}\~oes futuras para implementar sensores adicionais, desenvolver m\'etodos de ajuste autom\'atico dos par\^ametros fuzzy e aperfei\c{c}oar o julgamento de curvas.

% --- Referências ---
\bibliographystyle{sbc}
\bibliography{referencias}

\end{document}
